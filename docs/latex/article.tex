\title{YAMLine Specification}
\author{Roman Nevezhyn \\ 
%% Роман Невежин\\
 \underline{Ukraine}}
\date{\today}

\documentclass[14pt]{article}
\usepackage[a4paper, total={7in, 10in}]{geometry}
\usepackage{multicol}
\usepackage{polyglossia}   %% загружает пакет многоязыковой вёрстки
\setdefaultlanguage{russian}  %% устанавливает главный язык документа
    %\setdefaultlanguage[babelshorthands=true]{russian}  %% вместо предыдущей строки; доступны команды из пакета babel для русского языка
\setotherlanguage{english} %% объявляет второй язык документа
%%\defaultfontfeatures{Ligatures={TeX},Renderer=Basic}  %% свойства шрифтов по умолчанию. Для XeTeX опцию Renderer=Basic можно не указывать, она необходима для LuaTeX
%%\setmainfont[Ligatures={TeX,Historic}]{CMU Serif} %% задаёт основной шрифт документа
%%\setsansfont{CMU Sans Serif}                    %% задаёт шрифт без засечек
%%\setmonofont{CMU Typewriter Text}               %% задаёт моноширинный шрифт
\setmainfont{Arno Pro Display}
\setmonofont[Scale=0.9]{Ubuntu Mono}
%%\setmainfont{Nimbus Roman No9 L}
%%\setsansfont{Nimbus Sans L}
%%\setmonofont{Nimbus Mono L}
\tolerance=1300

\begin{document}
\maketitle
\begin{abstract}
This paper presents general YAML pipeline syntax \ldots
Мягкие булочки ✂✂✂
\end{abstract}
\begin{multicols}{2}
\section{Elements specifications}
YAML Pipeline is a set of YAML mappings that specify workflow.
There are following types of \verb|YAML elements|:
\begin{itemize}
  \item \verb|stage|:
    \begin{verbatim}
{
  "[name]": "str",
  "try": [],
  "[except]": [],
  "[else]": [],
  "[finally]": [],
  "[when]": "str"
}
}\end{verbatim}
  \item \verb|step|:
  \begin{verbatim}
{
  "[name]": "str",
  "strategy": "str",
  "[sets]": "str",
  "[args]": [],
  "[kwargs]": {"str": object},
  "[when]": "str"
}
  \end{verbatim}
  \item \verb|pipeline|:
  \begin{verbatim}
{
  "[aliases]": "str",
  "[import]": ["str"],
  "[metadata]": {"str": object},
  "[values]": {"str": object},
  "[name]": "str",  
  "try": [],
  "[except]": [],
  "[else]": [],
  "[finally]": [],
  "[when]": "str"
}
  
  [] = ["stage", "step"]
  \end{verbatim}
\end{itemize}

\section{General YAML Pipeline info}\label{general_info}
\paragraph{pipeline variables}Pipeline element manages variables that were setted in steps via \verb|sets: str| key value pair. \verb|str| represented in special varible syntax \texttt{\{\{ varible name \}\}} strictly matching folowing regexp \verb|(?<={{).*(?=}})|. Pipeline variables may be sent to executing step as a \verb|args| and \verb|kwargs| fields. All variables are have a global pipeline scope. If two steps set same variable name then the last step will override value that was set before.

\paragraph{name}Every \verb|YAML pipeline element| has the optional \verb|name: str| mapping. This field is intednet to be used as an identifier for import or as a description.

\paragraph{when}
Every \verb|YAML pipeline element| has the optional \verb|when: str| mapping. An element containig \verb|when: str| key value pair will only be executed if Python call \verb|bool(eval(str))| call will return \verb|True|. ~ref{pipeline variables} may be used in evaluation e. g.: \verb|when: len({{ var_name }}) > 10|. Variables name will be changed by parsing mechanism that above expression will be evalueated as: \verb|len(var_name) > 10| strictly matching folowing regexp \verb|{{.*}}|.

\section{Step}\label{step_info}
\paragraph{strategy}
A string in URI syntax that specifies function (callable) e.g.: \verb|strategy://strategies/database/experience_ids|

\paragraph{strategy}
A string in URI syntax: \verb|strategy://strategies/database/experience_ids|

\paragraph{sets}
Single variable name in pipeline varibales syntax e.g.: \verb|{{ var_name }}|

\paragraph{args}
An array of values and/or pipeline variables e.g.: 
\begin{verbatim}
	args:
	- {{ var_name1 }}
	- {{ var_name2 }}
	- 543
	- Hallo World 
\end{verbatim}

\paragraph{kwargs}
A mapping of key values pairs and values may be concrete values and/or pipeline variables e.g.:
\begin{verbatim}
      kwargs:
        db_columns_to_match: {{ var_name }}
        db_metadata_to_match:
          bots_count: 24
          arenas: 2
          pvp_bots: false
          groups: 1
\end{verbatim}

\section{Stage}\label{stage_info}
\paragraph{try}
\paragraph{except}
\paragraph{else}
\paragraph{finally}

\section{Pipeline}\label{pipeline_info}
\paragraph{aliases}
\paragraph{import}
\paragraph{metadata}
\paragraph{values}


\section{Aliases system}\label{aliases system}
There is a posibility to assign an alias for any \verb|YAML element|.
To specify an alias for the \verb|YAML element| one should create \verb|.yaml| file with following mapings:

\section{Results}\label{results}
In this section we describe the results.

\section{Conclusions}\label{conclusions}
We worked hard, and achieved very little.

\bibliographystyle{abbrv}
\bibliography{main}
\end{multicols}
\end{document}
This is never printed